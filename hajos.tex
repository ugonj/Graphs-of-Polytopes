\documentclass[a4paper,12pt]{article}
\usepackage{mathtools,amsthm}
\usepackage{tikz}
\usepackage{tkz-graph}
\usepackage{tkz-berge}
\usepackage[autostyle]{csquotes}
\usepackage[backend=biber]{biblatex}
\addbibresource{hajos.bib}


\newtheorem{theorem}{Theorem}
\newtheorem{lemma}{Lemma}
\newtheorem{conjecture}{Conjecture}

\title{Colouring of graphs of polytopes}
\author{Guillermo Pineda Villavicencio and Julien Ugon}

\begin{document}
  \maketitle

  In this document we report on the results of experiments on the validity of Haj\'os' and Thomson's conjectures on graphs of polytopes.

  \section{Haj\'os' conjecture}
  Haj\'os' conjecture states that:

  \begin{conjecture}
  \end{conjecture}


  \subsection{Small  graphs}

  We tested Haj\'os' conjecture on the graph of all the polytopes provided by \textcite{matroids}. Namely, we  applied the following  algorithm:

  \begin{enumerate}
    \item Compute the  graph of the polytope;
    \item Compute the chromatic number \(t\) of the graph;
    \item Verify  that \(K_t\) is a topological minor of the graph.
  \end{enumerate}

  We implemented this algorithm in Sage \parencite{SteinJoyner2005}. The code is available in \parencite{repo}.

  \subsection{Catlin's Counter examples}

  In this section we consider  \textcite{catlin1979hajos}'s famous counter-example to Haj\'os' conjecture. In this paper, Catlin gives a family of graphs with chromatic number  \(n\) that contain no  subdivided \(K_n\) as a  subgraph. Here we show that the smallest graph in this family (namely \(L(_3C_5)\) ) is not the  graph of a polytope.

  \begin{figure}
  \input{lm35.tikz}
  \caption{\(L(_3C_5))\)}
  \label{fig:L3C5}
  \end{figure}

  \begin{lemma}
    \(L(_3C_5)\) is not the graph of a polytope.
  \end{lemma}

  \begin{proof}

  First note that \((L(_3C_5)\) is not planar, and so cannot be the graph of a 3-polytope. Since it is not 5-connected (the  vertices \(\{1,2,9,10\}\) disconnect the graph, the  dimension is at most 4.), it cannot be the graph of a \(d\)-polytope for \(d\geq 5\). It remains the case \(d=4\).

  The graphs of the facets of any 4-polytope must be planar, 3-connnected, and should not disconnect the entire graph. Let us consider those that contain Vertex \(3\) and either of vertices  1,2 and 9. The graphs of these facets must be one of the following subgraphs of \(L(_3C_5)\):

  \[
  \begin{array}{ccc}
    1 & 2 & 9\\
    \{1,2,3,4\} & \{1,2,3,4\} & \{3,4,9,10\}  \\
    \{1,2,3,5\} & \{1,2,3,5\} & \{3,5,9,10\}  \\
    \{1,3,4,5,9\} & \{2,3,4,5,9\} & \{1,3,4,5,9\}  \\
    \{1,3,4,5,10\} & \{2,3,4,5,10\} & \{2,3,4,5,9\}  \\
    \{1,3,4,5\} & \{2,3,4,5\} & \{3,4,5,9\} 
  \end{array}
  \]

  The edge \(\{1,3\}\) must belong to at least three facets amongs the candidates: \(\{1,2,3,4\}\),\(\{1,2,3,5\}\),\(\{1,3,4,5,9\}\),\(\{1,3,4,5,10\}\), and \(\{1,3,4,5\}\). At least one of these three facets must contain the  vertices  \(\{3,4,5\}\). By a similar argument, the edge \(\{2,3\}\) must also belong to  a facet containing the vertices \(\{3,4,5\}\). The intersection between these two facets is either the triangle \(\{3,4,5\}\), or contains the four vertices \(\{3,4,5,k\}\) for \(k\in \{9,10\}\). We can rule out the latter case, since this would require the graph of this 2-face to be complete.

  We can conclude that the sets of vertices \(\{1,3,4,5\}\) and \(\{2,3,4,5\}\) both form a facet of the polytope.

  %Each of the edges \(\{1,3\}\)  and \(\{2,3\}\) must belong to at least three facets. For each of these edges, this implies that one of these facets must contain the vertices \(\{3,4,5\}\), and  the intersection between these two facets is precisely the ridge (triangle) \(\{3,4,5\}\). The only possibility for this to happen is that the facets are \(\{1,3,4,5\}\)  and \(\{2,3,4,5\}\). Indeed, all  the  other  subgraphs containing the vertices \(\{3,4,5\}\) are the graphs of  a bipyramid with base \(\{3,4,5\}\).

  The edge\(\{3,9\}\) must also belong to at least three facets, one of which contains the ridge \(\{3,4,5\}\). This cannot happen, since this would imply  that the ridge  is contained in three distinct facets.

  \end{proof}

  \section{Thompson's conjecture}

   Abigail Thompson made the following conjecture~\parencite{kal14}:

  \begin{conjecture}
    Graphs of simple \(d\)-polytopes with an even number of vertices are \(d\)-colourable.
  \end{conjecture}

  We tested these conjecture of small polytopes provided by~\textcite{matroids}, using the following algorithm. Namely, we selected all simplicial polytopes in this collection with an even number of facets. We then verified that the chromatic numbers of each of these polytopes is at most \(d\).

  The code is available in \parencite{repo}

  \printbibliography

\end{document}
